\documentclass[12pt,openright,oneside,a4paper,english,brazil]{iaeMM}

% usar a opção "oficial" para os nomes oficiais do ministério, comando, instituto, e o logo oficial do instituto.
%\documentclass[12pt,openright,oneside,a4paper,english,brazil,oficial]{iaeMM}

% ----------------------------------
% Definindo o layout do documento

% ----------------------------------

% ----
% Início do documento
% ----
\begin{document}

%-----------------------------------
% Definições do documento


\def\Titulo{Influência do Coronavírus e da batida da asa da borboleta no sul da Ucrânia.}

\def\CodIAE{123-000000/X0010}

\def\CodDiv{ABC-EF-001-2020}

\def\Divisao{DIVISÃO DE ESTUDOS DO BALDE DE ABOBRINHA}

\def\Origem{ABC}

\def\Aprovador{Adão \textbf{Fulano} de Tal - Pat. Esp.}

\def\FuncaoAprovador{Chefe da TRIBO}

\def\Revisor{Abel \textbf{Fulano} de Tal - Pat. Esp.}

\def\FuncaoRevisor{Chefe da TRIBO-OCA}

\def\Sigilo{Restrito}
\def\Volume{1}
\def\TotalVolumes{1}
\def\Data{01/04/2020} % Campo* DD/MM/AAAA.

\def\Autori{Caim \textbf{Fulano} de Tal - Pat. Esp.}
\def\FuncaoAutori{Chefe da TRIBO-OCA-REDE}


\def\Original{VDIR-GC-DT} % TODO: vai pra classe, dado que não muda!

\def\Copias{DIV1, DIV2, DIV3}

\def\PalavrasChave{Índio, Rede, Borboleta}

\def\Keywords{Indian, Hammock, Butterfly}

\def\PalavrasChaveIndexacao{Índio, Rede, Borboleta}

\def\ArquivoEletronico{123-000000-X0010.pdf}

\def\CodSIGTEC{}

\def\Observacoes{}

\def\Resumo{O presente documento apresenta cálculos preliminares para a turbulência do bater de asas da borboleta.}

\def\Abstract{The book is on the Table.}

\def\Projeto{BRB01}
\def\InicioCodigo{123} 
\def\nivelSuperior{Não há}
\def\nivelInferior{Não há}



%-----------------------------------
% Seleciona o idioma do documento (conforme pacotes do babel)

%\selectlanguage{english}
\selectlanguage{brazil}

% Retira espaço extra obsoleto entre as frases.
\frenchspacing 

\imprimircapa
\setcounter{page}{1}
\imprimirfolhaidentificacao
%\imprimirfolhadestinatarios

% ----------------------------------------------------------
% ELEMENTOS PRÉ-TEXTUAIS
% ----------------------------------------------------------
\setlength{\absparsep}{18pt} % ajusta o espaçamento dos parágrafos do resumo
\begin{resumo}[Resumo]
 \Resumo

 \textbf{Palavras-chave}: \PalavrasChave.
\end{resumo}% resumo em inglês
\begin{resumo}[Abstract]
 \begin{otherlanguage*}{english}
   \Abstract
 
   \noindent 
   \textbf{Keywords}: \Keywords.
 \end{otherlanguage*}
\end{resumo}\newpage
% ---
% inserir lista de ilustrações
% ---
\pdfbookmark[0]{\listfigurename}{lof}
\listoffigures*
\cleardoublepage
% ---

% ---
% inserir lista de tabelas
% ---
\pdfbookmark[0]{\listtablename}{lot}
\listoftables*
\cleardoublepage
% ---
% inserir o sumario
% ---
\pdfbookmark[0]{\contentsname}{toc}
\tableofcontents*
\cleardoublepage
% ---


% ----------------------------------------------------------
% ELEMENTOS TEXTUAIS
% ----------------------------------------------------------
\textual

% ---
% Corpo do Relatório
% ---

Ei Girardi.

Melhor a gente ei.

Por favor, gerar mais texto, imagens e figuras aleatórias para testar a classe.

Teste teste teste no overleaf.
%\input{./Chapter1/Chapter1}

% ----------------------------------------------------------
% ELEMENTOS PÓS-TEXTUAIS
% ----------------------------------------------------------
\postextual

% ----------------------------------------------------------
% Apêndices
% ----------------------------------------------------------
% ---
% Inicia os apêndices
% ---
\begin{apendicesenv}

%\input{./Appendix6/Appendix6.tex}

%\input{./Appendix7/Appendix7.tex}

%\input{./Appendix8/Appendix8.tex}

%\input{./Appendix9/Appendix9.tex}

%\input{./Appendix10/Appendix10.tex}

%\input{./Appendix11/Appendix11.tex}

\end{apendicesenv}

% ----------------------------------------------------------
% Referências bibliográficas
% ----------------------------------------------------------
\bibliography{Referencias}



\end{document}
